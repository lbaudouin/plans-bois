\documentclass[a4paper,oneside]{article}

\usepackage[frenchb]{babel}
\usepackage[utf8]{inputenc}  
\usepackage{graphicx}
\usepackage{url}
\usepackage{float}
\usepackage{color}
\usepackage{hyperref}
\usepackage{textcomp}
\usepackage{multicol}
\usepackage{subfig}
\usepackage{tikz}
\usetikzlibrary{patterns,snakes,shapes,calc,arrows,through,intersections}
   
\frenchbsetup{StandardLists=true}
\usepackage{enumitem}   
   
    
\usepackage{geometry}
\geometry{left=3cm,right=3cm,top=3cm,bottom=2cm}

\newcommand{\newtitle}{Chariot de marche}
\newcommand{\newauthor}{L\'eo \textsc{Baudouin}}


\begin{document}
%\maketitle
\vspace{-3cm}
\begin{center}
{\huge \newtitle}\\
\vspace{3mm}
\newauthor\\
\vspace{3mm}
\today\\
\vspace{3mm}
\url{http://lbaudouin.fr/bois.php}
\vspace{3mm}

\includegraphics[width=0.6\linewidth]{images/chariot.jpg}

\end{center}



\section{Fournitures}

\subsection{Matériel}

\begin{tabular}{l l}
\begin{minipage}{0.5\linewidth}

\textbf{Contreplaqué multi-pli :}
 \begin{itemize}[label=$\bullet$]
\item en 15~mm :
\begin{itemize}
\item 30*40~cm
\item 30*30~cm
\item 45*20~cm ($\times 2$)
\item 3*5~cm ($\times 10$)
\end{itemize}
\item en 10~mm :
\begin{itemize}
\item 30.7*12.5~cm
\end{itemize}
\end{itemize}

\textbf{Roues :}
\begin{itemize}
\item Roues fixe (125~mm) ($\times 4$)~\href{http://www.leroymerlin.fr/v3/p/produits/roue-fixe-sur-axe-pour-manutention-diametre-125-mm-e21277}{reférence}
\item Roues libres ($\times 2$)~\href{http://www.leroymerlin.fr/v3/p/produits/roulette-pivotante-a-platine-pour-ameublement-diametre-50-mm-e21285}{reférence}
\end{itemize}
\end{minipage}
&
\begin{minipage}{0.5\linewidth}
\textbf{Autre :}
\begin{itemize}
\item Tube acier $\sim$ 10~cm * 14~mm (intérieur 12~mm) ($\times 4$) ~\href{http://www.leroymerlin.fr/v3/p/produits/tube-rond-lisse-en-acier-brut-l1m-x-ep0-1cm-d1-4cm-e17176}{reférence}
\item Vis M10 * 12~cm ($\times 4$)
\item Écrou M10 avec frein ($\times 4$)
\item Vis fraisée 40~mm ($\times 16$)
\item Rondelle 30~mm (intérieur 15~mm) ($\times 4$)
\item Rondelle 20~mm (intérieur 11~mm) ($\times 8$)
\item Peintures
\item Vernis
\end{itemize}
\end{minipage}
\end{tabular}

\subsection{Prix}

\begin{tabular}{|l|c|}
\hline
Bois & $\sim$ 20\texteuro \\
Roues & $\sim$ 20\texteuro \\
Quincaillerie & $\sim$ 10\texteuro \\
Vernis & $\sim$ 5\texteuro \\
\hline
\hline
Total & $\sim$ 55\texteuro \\
\hline
\end{tabular}

\vspace{2mm}
Prix de vente neuf : $\sim$ 90\texteuro

\newpage

\section{Outillage}

\begin{multicols}{2}
\begin{itemize}
\item Serre-joints
\item Scie circulaire
\item Scie sauteuse
\item Scie cloche (30mm \& 50mm)
\item Embout à chanfreiner
\item Forêt à bois (1.5mm, 10mm \& 15mm)
\item Papier abrasif (grain 80 \& 120)
\item Pinceau pour vernis
\end{itemize}
\end{multicols}

\section{Pièces}
\subsection{En 15 mm d'épaisseur}
\begin{figure}[!h]
\captionsetup[subfigure]{labelformat=empty}
\subfloat[Pièce~1]{
\begin{tikzpicture}[scale=0.15]
    \draw[rounded corners=10] (0,0) -- (0,40) -- (30,40) -- (30,0);
    \draw (0,0) -- (30,0);
    \draw[color=gray,very thin,dashed] (4,36.5) circle (1.5); 
    \draw[color=gray,very thin,dashed] (26,36.5) circle (1.5);
    \draw (4,38) arc (90:270:1.5cm);
    \draw (26,38) arc (90:-90:1.5cm);
    \draw (4,38) -- (26,38);
    \draw (4,35) -- (26,35);
    
    \draw[<->] (0,-2) -- (30,-2) node [midway, below] {30cm};
    \draw[<->] (-2,0) -- (-2,40) node [ rotate=90, midway, above] {40cm};
    
    
    \draw[color=gray,very thin,dashed] (-3,36.5) -- (4,36.5);
    \draw[color=gray,very thin,dashed] (4,42) -- (4,36.5);
    \draw[<->] (-3,36.5) -- (-3,40) node [ rotate=90, midway, above] {3.5cm};    
    \draw[<->] (0,42) -- (4,42) node [midway, above] {4cm};    
    
    
    \draw[<->] (32,38) -- (32,35) node [ rotate=90, midway, below] {3cm};
  \end{tikzpicture}
  }
\subfloat[Pièce~2]{
\begin{tikzpicture}[scale=0.15]
    \draw (0,0) rectangle (30,30);
      
    \draw[<->] (0,-2) -- (30,-2) node [midway, below] {30cm};
    \draw[<->] (-2,0) -- (-2,30) node [ rotate=90, midway, above] {30cm};
  \end{tikzpicture}
  }
  
\subfloat[Pièce~3]{
\begin{tikzpicture}[scale=0.15]
    \draw[color=gray,very thin,dashed] (0,0) rectangle (45,20);
    
    \draw[name path=arc1] (0,10) arc (0:-90:-10cm);
	\draw (0,3) -- (0,10);
    \draw (0,3) arc (180:270:3cm);
	\draw (3,0) -- (42,0);
    \draw (42,0) arc (270:360:3cm);
    \draw (10,20) .. controls (25,20) and (45,10) .. (45,3);
    
    
    \path[name path=line1] (5,3) -- (5,20);
    \path[name path=line2] (6.1,3) -- (6.1,20);
    
    \path [name intersections={of=arc1 and line1}];
    
	\draw (intersection-1) -- (5,3);
	\draw (5,3) -- (6.1,3);
	
	 \path [name intersections={of=arc1 and line2}];
	 
	\draw (intersection-1) -- (6.1,3);
    
    
    \draw (14,14) circle (2.5);
    \fill (14,14) circle (0.1) node[below] {\tiny (14,14)};
    \draw[<->] (11,11) -- (16.5,11) node [midway, below] {\tiny 5cm};
    
	%vis
    \fill (11,3.5) circle (0.2) node[above] {\tiny (11,3.5)};
    \fill (27,3.5) circle (0.2) node[below] {\tiny (27,3.5)};
    \fill (31,6.5) circle (0.2) node[below] {\tiny (31,6.5)};
    \fill (33,12.5) circle (0.2) node[below] {\tiny (33,12.5)};
	    
	    
    \draw (6.5,1.5) circle (0.75) node[right] {\tiny (6.5,1.5)};
    \draw (40,1.5) circle (0.75) node[above] {\tiny (40,1.5)};
      
    \draw[<->] (0,22) -- (5,22) node [midway, above] {5cm};  
    \draw[<->] (5,22) -- (6,22) node [midway, below] {\tiny 11mm};       
      
    \draw[<->] (0,-2) -- (45,-2) node [midway, below] {45cm};
    \draw[<->] (-2,0) -- (-2,20) node [ rotate=90, midway, above] {20cm};
    \draw[<->] (4.5,0) -- (4.5,3) node [ rotate=90, midway, above] {\tiny 3cm};
  \end{tikzpicture}
  }
\subfloat[Pièce~4]{
\begin{tikzpicture}[scale=0.15]
    \draw (0,0) rectangle (5,3);
    
    \fill (1,1.5) circle (0.2);
    \fill (4,1.5) circle (0.2);
    
    \draw (2.5,1.5) circle (0.75);
    
    \draw[<->] (0,4) -- (1,4) node [midway, above] {\tiny 1cm};
    \draw[<->] (4,4) -- (5,4) node [midway, above] {\tiny 1cm};
    \draw[<->] (6,0) -- (6,1.5) node [ rotate=90, midway, below] {\tiny 1.5cm};
      
    \draw[<->] (0,-2) -- (5,-2) node [midway, below] {5cm};
    \draw[<->] (-2,0) -- (-2,3) node [ rotate=90, midway, above] {3cm};
  \end{tikzpicture}
  }
\end{figure}

\subsection{En 10 mm d'épaisseur}
\begin{figure}[!h]
\captionsetup[subfigure]{labelformat=empty}
\subfloat[Pièce~5]{
\begin{tikzpicture}[scale=0.1]
    \draw (0,0) rectangle (30.7,12.5);
      
    \draw[<->] (0,-2) -- (30.7,-2) node [midway, below] {30.7cm};
    \draw[<->] (-2,0) -- (-2,12.5) node [ rotate=90, midway, above] {12.5cm};
  \end{tikzpicture}
  }
\end{figure}

\newpage

\section{Fabrication}

\begin{enumerate}
\item Découper les planches~1 et~2 aux dimensions ci-dessus à l'aide d'une scie circulaire.
Faire attention à ce que les deux planches aient exactement la même largeur (30~cm).
\item Dans la pièce~1, percer les trous de 30~mm avec une scie cloche, finir l'ouverture avec une scie sauteuse, ou une scie à chantourner.
\item Adoucir les coins supérieurs de la pièce~1.
\item Reporter le contour de la pièce~3 sur deux nouvelles planches.
Découper le contour à la scie à ruban, ou à la scie sauteuse.
Dans ce dernier cas, afin d'avoir exactement la même forme, il est possible de maintenir les deux planches 3 ensemble à l'aide de serre-joints.
\item Percer les trous de 50~mm à la scie cloche.
\item Faire une rainure de 11~mm dans les deux pièces~3 (vers l'intérieur du chariot) à la défonceuse en déplaçant progressivement le guide linéaire, ou à la scie circulaire (dans ce cas il faut faire la rainure sur toute la hauteur).
\item Percer les pré-trous en 1.5~mm, puis les chanfreiner (depuis l'extérieur).


\item Coller [et/ou visser] les rectangles (pièces~4) à l'intérieur des pièces~3.

\begin{figure}[!h]
\captionsetup[subfigure]{labelformat=empty}
\subfloat[Placement des pièces~4]{
\begin{tikzpicture}[scale=0.15]
    \draw[color=gray,very thin,dashed] (0,0) rectangle (45,20);
    
    \draw[name path=arc1] (0,10) arc (0:-90:-10cm);
	\draw (0,3) -- (0,10);
    \draw (0,3) arc (180:270:3cm);
	\draw (3,0) -- (42,0);
    \draw (42,0) arc (270:360:3cm);
    \draw (10,20) .. controls (25,20) and (45,10) .. (45,3);    
    
    \draw (37.5,0) rectangle +(5,3);
    \draw (4,0) rectangle +(5,3);

  
    \draw[gray] (40,1.5) circle (0.75);
    \draw[gray] (6.5,1.5) circle (0.75);
      
    \draw[<->] (0,22) -- (45,22) node [midway, above] {45cm};
    \draw[<->] (0,-1) -- (4,-1) node [midway, below] {4cm};
    %\draw[<->] (4,-1) -- (9,-1) node [midway, below] {5cm};
   % \draw[<->] (37.5,-1) -- (42.5,-1) node [midway, below] {5cm};
    \draw[<->] (42.5,-1) -- (45,-1) node [midway, below] {2.5cm};
    
    \draw (6.5,3) node[above] {$\times 3$};
    \draw (40,3) node[above] {$\times 2$};

  \end{tikzpicture}
  }
\end{figure}

\item Percer les 4~trous de 15~mm à l'aide d'une perceuse à colonne en commençant par le coté des pièces~4 afin d'être parfaitement au centre.
\item Découper 4~bouts de tube, pour la longueur ($x$) reportez vous au schéma suivant :

\begin{figure}[!h]
\captionsetup[subfigure]{labelformat=empty}
\subfloat[Longueur du tube ($x$)]{
\begin{tikzpicture}[scale=0.5]
   \pgfmathsetmacro{\wrondelle}{0.1}%
\pgfmathsetmacro{\hrondelle}{2}%
\pgfmathsetmacro{\erondelle}{0.4}%
\pgfmathsetmacro{\xrondelle}{-\wrondelle}%

\pgfmathsetmacro{\xrondellebis}{4.5}%
\pgfmathsetmacro{\erondellebis}{0.5}%
\pgfmathsetmacro{\wrondellebis}{0.2}%
\pgfmathsetmacro{\wrondellebis}{0.2}%
\pgfmathsetmacro{\hrondellebis}{2.6}%
\pgfmathsetmacro{\xroue}{\xrondellebis+\wrondellebis}%
\pgfmathsetmacro{\wroue}{4}%
\pgfmathsetmacro{\hroue}{12}%

\pgfmathsetmacro{\xrondelleter}{\xroue+\wroue}%
\pgfmathsetmacro{\wrondelleter}{0.1}%
\pgfmathsetmacro{\hrondelleter}{2}%
\pgfmathsetmacro{\erondelleter}{0.5}%

\pgfmathsetmacro{\xtvis}{\xrondelleter+\wrondelleter}%
\pgfmathsetmacro{\wtvis}{1}%
\pgfmathsetmacro{\htvis}{1.5}%


\pgfmathsetmacro{\xvis}{-1.5}%
\pgfmathsetmacro{\wvis}{\xtvis-\xvis}%
\pgfmathsetmacro{\hvis}{0.8}%

\pgfmathsetmacro{\xtube}{0}%
\pgfmathsetmacro{\etube}{0.1}%
\pgfmathsetmacro{\wtube}{4.5+\wrondellebis+\wroue-0.2}%
\pgfmathsetmacro{\htube}{1.4}%


    \filldraw (\xrondelle,-\hrondelle/2)  rectangle +(\wrondelle,\hrondelle);
	\filldraw[white,draw=black] (\xrondelle,-\hrondelle/2+\erondelle)  rectangle +(\wrondelle,\hrondelle-2*\erondelle);

    %Pièces 4
	\draw[pattern=north west lines] (0,-1.5) rectangle +(1.5,3);
	 \filldraw[fill=white, draw=black] (0,0.8) rectangle +(1.5,-1.6);
	\draw [pattern=north east lines](1.5,-1.5) rectangle +(1.5,3);
	 \filldraw[fill=white, draw=black] (1.5,0.8) rectangle +(1.5,-1.6);
	\draw[<-] (0.75,2) -- (0.75,3) node [rotate=90,above] {\tiny Pièce~4}; 
	\draw[<-] (2.25,2) -- (2.25,3) node [rotate=90,above] {\tiny Pièce~4}; 
	
	%Pièce 3
    \draw[pattern=north west lines] (3,1.5) -- (3,5) .. controls (4,6) and (3.5,4) .. (4.5,5) --  (4.5,-1.5) -- (3,-1.5);
	 \filldraw[fill=white, draw=black] (3,0.8) rectangle +(1.5,-1.6);
	\draw[<-] (3.75,5.5) -- (3.75,6.5) node [rotate=90,above] {\tiny Pièce~3}; 
    
    %rondelle
    \filldraw (\xrondellebis,-\hrondellebis/2)  rectangle +(\wrondellebis,\hrondellebis);
	\filldraw[white,draw=black] (\xrondellebis,-\hrondellebis/2+\erondellebis)  rectangle +(\wrondellebis,\hrondellebis-2*\erondellebis);
    
    %roue
    \draw[dashed,gray] (\xroue,-\hroue/2)  rectangle +(\wroue,\hroue);
	\draw (\xroue,-\hroue/2+1)  -- (\xroue,\hroue/2-1);
	\draw (\xroue+\wroue,-\hroue/2+1)  -- (\xroue+\wroue,\hroue/2-1);
    
    \draw (\xroue,-0.8)  rectangle +(\wroue,1.6);
    \draw[<->] (\xroue,-\hroue/2-0.2) -- (\xroue+\wroue,-\hroue/2-0.2) node [midway, below] {\tiny largeur roue};
    \coordinate (A) at (\xroue,0.8);
    \draw [pattern=north east lines] (\xroue,0.8)-- +(0,1)
		 .. controls (\xroue,2.5) and (\xroue+\wroue/4,2)
 .. (\xroue+\wroue/4,\hroue/4 ) 
.. controls (\xroue+\wroue/4,\hroue/4 + 1) and (\xroue,\hroue/2-2) 
..  (\xroue,\hroue/2-1)
.. controls (\xroue,\hroue/2) and (\xroue +\wroue/3-1,\hroue/2) ..
 (\xroue +\wroue/3,\hroue/2)--(\xroue +2*\wroue/3,\hroue/2)
 .. controls  (\xroue +2*\wroue/3+1,\hroue/2) and (\xroue + \wroue,\hroue/2)
 .. (\xroue+\wroue,\hroue/2-1)
.. controls (\xroue+\wroue,\hroue/2-2)  and (\xroue+3*\wroue/4,\hroue/4 + 1) 
..  (\xroue+3*\wroue/4,\hroue/4 )
.. controls  (\xroue+3*\wroue/4,2)  and (\xroue+\wroue,2.5)
.. (\xroue+\wroue,1.8) -- (\xroue+\wroue,0.8) -- (\xroue,0.8);

\draw [pattern=north east lines] (\xroue,-0.8) -- +(0,-1)
		 .. controls (\xroue,-2.5) and (\xroue+\wroue/4,-2)
 .. (\xroue+\wroue/4,-\hroue/4 ) 
.. controls (\xroue+\wroue/4,-\hroue/4 - 1) and (\xroue,-\hroue/2+2) 
..  (\xroue,-\hroue/2+1)
.. controls (\xroue,-\hroue/2) and (\xroue +\wroue/3-1,-\hroue/2) ..
 (\xroue +\wroue/3,-\hroue/2)--(\xroue +2*\wroue/3,-\hroue/2)
 .. controls  (\xroue +2*\wroue/3+1,-\hroue/2) and (\xroue + \wroue,-\hroue/2)
 .. (\xroue+\wroue,-\hroue/2+1)
.. controls (\xroue+\wroue,-\hroue/2+2)  and (\xroue+3*\wroue/4,-\hroue/4 - 1) 
..  (\xroue+3*\wroue/4,-\hroue/4 )
.. controls  (\xroue+3*\wroue/4,-2)  and (\xroue+\wroue,-2.5)
.. (\xroue+\wroue,-1.8) -- (\xroue+\wroue,-0.8) -- (\xroue,-0.8);
    

	\draw[<-] (\xroue+\wroue/2,\hroue/2+0.5) -- (\xroue+\wroue/2,\hroue/2+1.5) node [rotate=90,above] {\tiny Roue};     
    
    %rondelle
     \filldraw (\xrondelleter,-\hrondelleter/2)  rectangle +(\wrondelleter,\hrondelleter);
	\filldraw[white,draw=black] (\xrondelleter,-\hrondelleter/2+\erondelleter)  rectangle +(\wrondelleter,\hrondelleter-2*\erondelleter);

	%tube
    \filldraw[fill=red!40!white, draw=red] (\xtube,-\htube/2) rectangle +(\wtube,\htube);
	\fill[red] (\xtube,-\htube/2) rectangle +(\wtube,\etube);
	\fill[red] (\xtube,\htube/2) rectangle +(\wtube,-\etube);
	
    \draw[<->] (0,-3) -- (\wtube,-3) node [midway, below left] { $x$ cm};
    \draw[>-<] (\wtube-0.1,-3) -- (\xrondelleter+0.1,-3) node [midway, below right] {\tiny 1-2 mm};

	%ecrou
    \draw[pattern=north east lines] (\xrondelle-\wtvis,-\htvis/2) rectangle +(\wtvis,\htvis);

	%vis
    \draw[pattern=north west lines] (\xtvis,-\htvis/2) rectangle +(\wtvis,\htvis);

	\filldraw[fill=white,draw=black,snake=zigzag, segment amplitude=1,segment length=3] (\xvis,-\hvis/2) -- (\xvis+\wvis,-\hvis/2) [snake=none] -- (\xvis+\wvis,\hvis/2)  [snake=zigzag, segment amplitude=1,segment length=3] -- (\xvis,+\hvis/2)  [snake=none] -- (\xvis,-\hvis/2) ;
	\fill[pattern=north west lines,snake=zigzag, segment amplitude=1,segment length=3] (\xvis,-\hvis/2) -- (\xvis+\wvis,-\hvis/2) [snake=none] -- (\xvis+\wvis,\hvis/2)  [snake=zigzag, segment amplitude=1,segment length=3] -- (\xvis,+\hvis/2)  [snake=none] -- (\xvis,-\hvis/2) ;
    
    
    
  \end{tikzpicture}
  }
\end{figure}

\item Assembler les différentes pièces en vissant un coté puis l'autre.
\item Mesurer l'écartement entre les rainures et ajuster les dimensions de la pièce~5 pour qu'elle puisse coulisser facilement (1~mm de marge de chaque coté).
La découper à la scie circulaire.
\item Adoucir tous les bords au papier abrasif.
\item[Facultatif.] Ajouter une touche de peinture puis vernir.
\end{enumerate}


\subsection{Conseils}
\begin{itemize}
\item Afin d'éviter que le bois n'éclate, penser à utiliser des planches sacrificielles (surtout pour les trous).
\item Toujours maintenir la planche bien plaqué contre son support à l'aire de \href{http://www.leroymerlin.fr/v3/p/produits/serre-joint-1-main-dexter-300-mm-e51423}{serre-joints}.
\item Utiliser un rail de guidage pour toutes les lignes droites (une \href{http://www.leroymerlin.fr/v3/p/produits/regle-avec-embout-nespoli-150-cm-e67206}{règle} avec des \href{http://www.leroymerlin.fr/v3/p/produits/lot-de-serre-joints-a-cadre-5-mm-e51365}{serre-joints} suffisent).
\end{itemize}

\subsection{Consignes}


\section{Utilisation}

Pour l'apprentissage de la marche ne pas mettre les roues folles, le chariot doit aller uniquement en ligne droite.
Penser à ajuster le freinage à l'aide des boulons sur les différentes roues.

Une fois que l'enfant marche correctement, ajouter les roues folles pour plus de liberté.


\vfill
\hfill \small \href{https://github.com/lbaudouin/plans-bois/blob/master/chariot.tex}{Document} réalisé avec \LaTeX
\end{document}